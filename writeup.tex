\documentclass{article}

\usepackage{titlesec}
\usepackage{titling}

\titleformat{\section}
{\Large\bfseries}
{}
{0em}
{}[\titlerule]

\titleformat{\subsection}
{\large\bfseries}
{}
{0em}
{}

\titleformat{\subsubsection}
{\bfseries}
{}
{0em}
{}

\begin{document}

\title{Project: Kinematics Pick \& Place}
\author{Brett Gleason}

\section{Rubric}
% [Rubric](https://review.udacity.com/#!/rubrics/972/view) Points
Here I will consider the rubric points individually and describe how I addressed each point in my implementation.  

\section{Writeup / README}

\subsection{1. Provide a Writeup / README that includes all the rubric points and how you addressed each one.  You can submit your writeup as markdown or pdf.}

You're reading it!

\section{Kinematic Analysis}
\subsection{1. Run the forward\_kinematics demo and evaluate the kr210.urdf.xacro file to perform kinematic analysis of Kuka KR210 robot and derive its DH parameters.}

Using the model of the Kuka KR210 robotic arm in the forward kinematics demo as well as the description of the joints within the URDF file, a schematic diagram of the robot can be drawn.

fig. 1. Basic schematic as shown in project lesson 10

Next the joints are labeled from 1 to n and the links are labeled from 0 to n.

fig. 2. Schematic showing joint and link numbers as shown in project lesson 10

After the joints and links are labeled, reference frames can be defined for each joint.

fig. 3. Schematic showing reference frames for each joint as shown in project lesson 10

Using the reference frames the Denavit-Hartenberg parameters can be defined. For this project the DH parameters are defined using the convention described by John J. Craig in his book Introduction to Robotics: Mechanics and Control. The definitions are as follows (from lesson 2 section 12):

Twist angle (alpha sub(i-1)): angle between Zhat sub(i-1) and Zhat sub(i) measured about Xhat sub(i-1) in a right hand sense.
Link length (a sub(i-1)): distance from Zhat sub(i-1) to Zhat sub(i) measured along Xhat sub(i-1).
Link offset (d sub(i)): signed distance from Xhat sub(i-1) to Xhat sub(i) measured along Zhat sub(i).
Joint angle: angle between Xhat sub(i-1) and Xhat sub(i) measured about Zhat sub(i) in a right hand sense.

Table 1. Denavit-Hartenberg parameter table with values derived from the URDF file

\subsection{2. Using the DH parameter table you derived earlier, create individual transformation matrices about each joint. In addition, also generate a generalized homogeneous transform between base\_link and gripper\_link using only end-effector(gripper) pose.}

% Here's | A | Snappy | Table
% --- | --- | --- | ---
% 1 | `highlight` | **bold** | 7.41
% 2 | a | b | c
% 3 | *italic* | text | 403
% 4 | 2 | 3 | abcd

\subsection{3. Decouple Inverse Kinematics problem into Inverse Position Kinematics and inverse Orientation Kinematics; doing so derive the equations to calculate all individual joint angles.}

% And here's another image! 


\section{Project Implementation}

\subsubsection{1. Fill in the `IK\_server.py` file with properly commented python code for calculating Inverse Kinematics based on previously performed Kinematic Analysis. Your code must guide the robot to successfully complete 8/10 pick and place cycles. Briefly discuss the code you implemented and your results.}


Here I'll talk about the code, what techniques I used, what worked and why, where the implementation might fail and how I might improve it if I were going to pursue this project further.  


And just for fun, another example image:
% ![alt text][image3]

\end{document}
